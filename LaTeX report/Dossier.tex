% Document class
\documentclass[titlepage]{report}

% Packages
\usepackage{lipsum,eurosym,enumitem,multicol,hyperref,geometry,graphicx,float,listings}
\usepackage[utf8]{inputenc}
\usepackage[T1]{fontenc}
\usepackage[dvipsnames]{xcolor}
\usepackage[french]{babel}
\usepackage{subfig}
\usepackage{listings}
\usepackage{qrcode}
\usepackage{ragged2e}

% Document geometry
\geometry{paper=a4paper,textwidth=16cm}

% Definition of \maketitle
\title{Projet de Programmation Java : MoonQuest}
\author{Par : Lucien Piat \\ \vspace{.5cm} Sous la direction de : Karkar Slim}
\date{Avril 2024}

\makeatletter         
\def\maketitle{
\raggedright
\begin{center}
\includegraphics[width = 0.5\textwidth]{img/logo_ub.png}\\[15ex]
{\Huge \bfseries \@title }\\[10ex] 
{\huge  \@author}\\[10ex] 
{\Large \@date}\\
\end{center}}
\makeatother


% Debut du Document
\begin{document}
\maketitle
\thispagestyle{empty} 
\newpage
\tableofcontents
\thispagestyle{empty} 

%Debut du sujet 
\newpage
\pagenumbering{arabic}
\justify 
\section{Introduction}

Java est un language de programmation orienté objet qui permet de produire des applications sécurisé et 
offre une portabilité avancée du code grace a une machine virtuelle.\\

Lors de ce projet de programmation, nous développerons le jeu "MoonQuest" grace à cet outil.  

\section{Analyse du Sujet}

Le jeu MoonQuest est présenté sous la forme d'un échiquier sur lequel des pieces sont placées.
Ces dernières sont des candidates parfaites pour êtres codées par des classes en Java.
Ainsi, nous essaieront de réduire au maximum les répétitions de code en créant un maximum de sous classes pour généraliser les objets manipulés.\\

Le plateau en lui même est pourra être une classe contenant toutes les cases. Ces cases pourront stocker les différentes pieces du plateau.\\

De nombreuses règles sont présentes dans le jeu, elle seront implémentées sous forme de méthodes() qui seront contenues dans les sous classes.
l’utilisation de super() nous permettra de moduler la réponse des méthodes en fonction de la sous classe utilisée.\\

Un des enjeux est de faire apparaître le plateau a l'écran. En effet, il faudra afficher les différentes cases et leurs contenu a chaque tour.
D'autre part, nous aurons besoins des menus pour permettre au joueur d’interagie avec le programme.\\

Enfin, nous implémenterons dans une classe Main, toutes les boucles qui appellent les différentes méthodes pour le bon déroulement de la partie.\\

\noindent Ainsi, voici comment nous pouvons décortiquer le sujet :\\
\begin{itemize}[label=$\bullet$] 
    \item Les différentes pieces seront des Classes parentes
    \item Le plateau sera une classe contenant les pieces
    \item Les règles du jeu seront codées dans les méthodes des classes
    \item Une affichage aura lieu a chaque tour
    \item Une boucle principale se chargera d'appeler les méthodes d'affichage et de déplacement.
\end{itemize}

\section{Solutions Envisagées}

\subsection{Pour l’Affichage}
\subsection{Pour le Plateau}
\subsection{Pour les Pieces}
\subsection{Pour les Joueurs}

\section{Algorithmes Choisis}

\subsection{Pour l’Affichage}
\subsection{Pour le Plateau}
\subsection{Pour les Pieces}
\subsection{Pour les Joueurs}

\section{Conclusion}
Jar

\end{document}